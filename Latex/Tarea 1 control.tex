\documentclass[10pt]{article}
\usepackage{bookmark}
% Formato extenso: report

% Formato corto: article

% Esto es para que el LaTeX sepa que el texto está en español:
\usepackage[spanish]{babel}

\usepackage{amsmath, amsthm, amsfonts,amssymb}

% Bórrame si quieres:
\usepackage{multicol}

% Referencias
\usepackage{hyperref}

% Paquete para escribir código
\usepackage{listings}
\lstset{basicstyle=\footnotesize\ttfamily,breaklines=true}
\usepackage{alltt}

% Paquete para incluir imágenes
\usepackage{graphicx}

% Paquete para incluir varias imágenes en una
\usepackage{subfig}

% para poder fijar las imágenes ([H])
\usepackage{float}

% para agregar opciones al enumerate
\usepackage{enumerate}


% Teoremas
\newtheorem{thm}{Teorema}[section]
\newtheorem{cor}[thm]{Corolario}
\newtheorem{lem}[thm]{Lema}
\newtheorem{prop}[thm]{Proposición}
\theoremstyle{definition}
\newtheorem{defn}[thm]{Definición}
\theoremstyle{remark}
\newtheorem{rem}[thm]{Observación}
\theoremstyle{definition}
\newtheorem{prob}{Problema}
\numberwithin{equation}{prob}

% Calculus symbols
\newcommand{\pd}[2]{\frac{\partial #1}{ \partial #2}}   % First partial derivative command
\newcommand{\td}[2]{\frac{\mathrm{d} #1}{ \mathrm{d} #2}}
\newcommand{\pdd}[2]{\frac{\partial^2 #1}{ \partial #2 ^2}}   % Second partial derivative command
\newcommand{\pddc}[3]{\frac{\partial^2 #1}{ \partial #2 \partial #3}}   % Second partial derivative command

% Continuum mechanics & FEM symbols
\def\sca   #1{\mbox{\rm{#1}}{}}
\def\mat   #1{\mbox{\boldmath $\mathsf #1$}}
\def\vec   #1{\mbox{\boldmath $#1$}{}}
\def\ten   #1{\mbox{\boldmath $#1$}{}}
\def\ltr   #1{\mbox{\sf{#1}}}
\def\bltr  #1{\mbox{\sffamily{\bfseries{{#1}}}}}

% math operators and symbols
\DeclareMathOperator{\dive}{div}
\DeclareMathOperator{\trace}{trace}
\DeclareMathOperator{\tr}{tr}
\DeclareMathOperator{\symm}{symm}
\DeclareMathOperator{\sk}{skew}
\DeclareMathOperator{\grad}{grad}
\DeclareMathOperator{\Grad}{Grad}
\DeclareMathOperator{\curl}{curl}
\DeclareMathOperator{\Curl}{Curl}
\def\R{\mbox{\(\mathbb{R}\)}}
\def\dx{\mbox{\(\,\mathrm{d}x\)}}


\usepackage{geometry}
\geometry{left=2.5cm, right=2.5cm, top=2cm, bottom=3cm}

\usepackage{makeidx}
\makeindex


\begin{document}

\begin{sloppypar}

\end{sloppypar}

\begin{titlepage}
	%%%%% NO MODIFICAR
	\begin{figure}
		\begin{minipage}{4cm}
			\includegraphics[width=0.9\textwidth]{./figures/logo}
		\end{minipage}
		\begin{minipage}{11cm}
			\vspace{4mm}
			{\sc UNIVERSIDAD DE VALPARAÍSO}\\
			Escuela de Ingeniería Civil Biomédica\\
			{\bf CBM422 - Ingeniería de control automático}\\
			\vspace{0mm}
			\hrulefill
		\end{minipage}
	\end{figure}
	\phantom{""}\vspace{60mm}


	%%%%% MODIFICAR
	\begin{center}
		\Huge{\textbf{Tarea 1}}\vspace{95mm}\\
		\raggedleft \Large{Jorge Gonzalo Alejandro Alcaíno Brevis}\\
		\raggedleft \Large{Maximiliano Antonio Gaete Pizarro}\\
	\end{center}


\end{titlepage}

\index{entry}


\section{Problema 1: Encuentre las transformadas inversas de Laplace, de las siguientes funciones}

\begin{equation}
	\begin{aligned}
		\text{(a)} \quad F_1(s) & = \frac{6s + 3}{s^2}              \\
		\text{(b)} \quad F_2(s) & = \frac{5s + 2}{(s + 1)(s + 2)^2}
	\end{aligned}
\end{equation}


\subsection{Parte (a):}
\[
	F_1(s) = \frac{6s + 3}{s^2}
\]
Descomponiendo la función:
\[
	F_1(s) = 6 \cdot \frac{s}{s^2} + 3 \cdot \frac{1}{s^2}
\]
La transformada inversa de Laplace de cada término es:
\[
	\mathcal{L}^{-1}\left\{ \frac{s}{s^2} \right\} = 1, \quad \mathcal{L}^{-1}\left\{ \frac{1}{s^2} \right\} = t
\]
Por lo tanto, la transformada inversa de \( F_1(s) \) es:
\[
	\mathcal{L}^{-1}\left\{ F_1(s) \right\} = 6 + 3t
\]

\subsection{Parte (b):}
\[
	F_2(s) = \frac{5s + 2}{(s+1)(s+2)^2}
\]
Descomponiendo en fracciones parciales:
\[
	F_2(s) = \frac{-3}{s+1} + \frac{3}{s+2} + \frac{8}{(s+2)^2}
\]
Para esta fracción racional más compleja, podemos aplicar descomposición en fracciones parciales. La forma general sería:
\[
	\frac{5s + 2}{(s + 1)(s + 2)^2} = \frac{A}{s+1} + \frac{B}{s+2} + \frac{C}{(s+2)^2}
\]
Primero, multiplicamos ambos lados por \( (s + 1)(s + 2)^2 \) para eliminar los denominadores:
\[
	5s + 2 = A(s + 2)^2 + B(s + 1)(s + 2) + C(s + 1)
\]

Resolviendo en Python
\begin{lstlisting}[language=Python]
    from sympy import symbols, Eq, solve

    # Definimos la variable s y los coeficientes A, B, C
    s = symbols('s')
    A, B, C = symbols('A B C')
    
    # Expresion original
    lhs = 5*s + 2
    
    # Expansion en fracciones parciales
    rhs = A*(s+2)**2 + B*(s+1)*(s+2) + C*(s+1)
    
    # Expandimos el lado derecho
    expanded_rhs = rhs.expand()
    
    # Igualamos ambos lados para resolver el sistema de ecuaciones
    equations = Eq(lhs, expanded_rhs)
    
    # Resolvemos para A, B y C
    coefficients = solve(equations, [A, B, C])
    coefficients
\end{lstlisting}

Expandiendo ambos lados y resolviendo para \( A \), \( B \), y \( C \), encontramos los coeficientes que necesitamos. Esto nos da la forma correcta para aplicar la transformada inversa de Laplace a cada término:
\[
	\frac{5s + 2}{(s+1)(s+2)^2} = \frac{-3}{s+1} + \frac{3}{s+2} + \frac{8}{(s+2)^2}
\]
Aplicando la transformada inversa de Laplace a cada término:
\[
	\mathcal{L}^{-1}\left\{ \frac{1}{s+1} \right\} = e^{-t}, \quad \mathcal{L}^{-1}\left\{ \frac{1}{s+2} \right\} = e^{-2t}, \quad \mathcal{L}^{-1}\left\{ \frac{1}{(s+2)^2} \right\} = t e^{-2t}
\]
Por lo tanto, la transformada inversa de \( F_2(s) \) es:
\[
	\mathcal{L}^{-1}\left\{ F_2(s) \right\} = -3 e^{-t} + 3 e^{-2t} + 8 t e^{-2t}
\]

\newpage

\section{Problema 2: Obtenga la Función de Transferencia del sistema definido por las ecuaciones}

\begin{equation}
	\begin{aligned}
		\begin{bmatrix}
			\dot{x}_1 \\
			\dot{x}_2 \\
			\dot{x}_3
		\end{bmatrix}
		 & =
		\begin{bmatrix}
			0  & 1  & 0  \\
			0  & 0  & 1  \\
			-2 & -4 & -6
		\end{bmatrix}
		\begin{bmatrix}
			x_1 \\
			x_2 \\
			x_3
		\end{bmatrix}
		+
		\begin{bmatrix}
			0 & 0 \\
			0 & 1 \\
			1 & 0
		\end{bmatrix}
		\begin{bmatrix}
			u_1 \\
			u_2
		\end{bmatrix}
		\\
		\begin{bmatrix}
			y_1 \\
			y_2
		\end{bmatrix}
		 & =
		\begin{bmatrix}
			1 & 0 & 0 \\
			0 & 1 & 0
		\end{bmatrix}
		\begin{bmatrix}
			x_1 \\
			x_2 \\
			x_3
		\end{bmatrix}
	\end{aligned}
\end{equation}

\subsection{Obtención de la función de transferencia del sistema:}

Dado el sistema de ecuaciones en el espacio de estados:

\[
	\dot{\mathbf{x}}(t) = A\mathbf{x}(t) + B\mathbf{u}(t)
\]
\[
	\mathbf{y}(t) = C\mathbf{x}(t) + D\mathbf{u}(t)
\]

donde las matrices \( A \), \( B \), \( C \) y \( D \) están definidas como:

\[
	A = \begin{bmatrix}
		0  & 1  & 0  \\
		0  & 0  & 1  \\
		-2 & -4 & -6
	\end{bmatrix}, \quad
	B = \begin{bmatrix}
		0 & 0 \\
		0 & 1 \\
		1 & 0
	\end{bmatrix}, \quad
	C = \begin{bmatrix}
		1 & 0 & 0 \\
		0 & 1 & 0
	\end{bmatrix}, \quad
	D = \begin{bmatrix}
		0 & 0 \\
		0 & 0
	\end{bmatrix}
\]

La función de transferencia \( \mathbf{G}(s) \) en el dominio de Laplace se calcula utilizando la fórmula:

\[
	\mathbf{G}(s) = C(sI - A)^{-1}B + D
\]

Paso 1: Calcular \( (sI - A) \)

La matriz identidad \( I \) es:

\[
	I = \begin{bmatrix}
		1 & 0 & 0 \\
		0 & 1 & 0 \\
		0 & 0 & 1
	\end{bmatrix}
\]

Por lo tanto, \( (sI - A) \) es:

\[
	sI - A = \begin{bmatrix}
		s & -1 & 0     \\
		0 & s  & -1    \\
		2 & 4  & s + 6
	\end{bmatrix}
\]

Paso 2: Inversa de \( (sI - A) \)

Ahora, calculamos la inversa de \( (sI - A) \):

Resolviendo en python
\begin{lstlisting}[language=Python]
    from sympy import Matrix, eye, symbols, simplify

    # Definir las matrices y la variable s
    s = symbols('s')
    A = Matrix([[0, 1, 0], [0, 0, 1], [-2, -4, -6]])
    B = Matrix([[0, 0], [0, 1], [1, 0]])
    C = Matrix([[1, 0, 0], [0, 1, 0]])
    D = Matrix([[0, 0], [0, 0]])
    I = eye(3)  # Matriz identidad de 3x3
    
    # Calcular (sI - A) y su inversa
    sI_A_inv = (s * I - A).inv()
\end{lstlisting}

\[
	(sI - A)^{-1} = \frac{1}{s^3 + 6s^2 + 4s + 2} \begin{bmatrix}
		s^2 + 6s + 4 & s + 6        & 1        \\
		2s + 4       & s^2 + 6s + 2 & s + 6    \\
		2            & 2s + 4       & s^2 + 6s
	\end{bmatrix}
\]

Paso 3: Multiplicación con \( B \)

Multiplicamos \( (sI - A)^{-1} \) por \( B \):

\[
	(sI - A)^{-1}B = \frac{1}{s^3 + 6s^2 + 4s + 2} \begin{bmatrix}
		1 & s + 6    \\
		s & s(s + 6)
	\end{bmatrix}
\]

Paso 4: Multiplicación con \( C \) y adición de \( D \)

Finalmente, multiplicamos por \( C \) y sumamos \( D \), obteniendo la función de transferencia:

Resolviendo en Python
\begin{lstlisting}[language=Python]
    # Calcular la funcion de transferencia G(s) = C(sI - A)^(-1)B + D
    G_s = simplify(C * sI_A_inv * B + D)
    G_s
\end{lstlisting}
\[
	\mathbf{G}(s) = \begin{bmatrix}
		\frac{1}{s^3 + 6s^2 + 4s + 2} & \frac{s + 6}{s^3 + 6s^2 + 4s + 2}    \\
		\frac{s}{s^3 + 6s^2 + 4s + 2} & \frac{s(s + 6)}{s^3 + 6s^2 + 4s + 2}
	\end{bmatrix}
\]

Esta es la función de transferencia del sistema, que describe la relación entre las entradas \( u_1 \) y \( u_2 \), y las salidas \( y_1 \) y \( y_2 \).

\newpage

\section{Problema 3: Mediante la simplificación del diagrama de bloques de la Figura 1, obtenga las siguientes funciones de transferencia}

\[
	\frac{Y(s)}{R(s)} \bigg|_{N=0}
	\quad \quad
	\frac{Y(s)}{N(s)} \bigg|_{R=0}
\]

\begin{figure}[h]
	\centering
	\includegraphics[width=0.9\textwidth]{./figures/Figura 1 ejercicio 3.png}
	\caption{Diagrama de bloques del sistema}
\end{figure}

\section{Problema 4: Obtenga la Función de Transferencia Eo(s)/Ei(s) del circuito eléctrico RLC que se muestra en
  la Figura 2}

\begin{figure}[h]
	\centering
	\includegraphics[width=0.6\textwidth]{./figures/Figura 2 ejercicio 4.png}
	\caption{Circuito eléctrico RLC}
\end{figure}


\section{Problema 5: Considere el sistema de la Figura 3(a). El factor de amortiguamiento relativo \texorpdfstring{$\zeta$}{}
  del sistema es igual a 0,158, y la frecuencia natural no amortiguada \texorpdfstring{$\omega_n$}{omega\_n} es de 3,16 rad/seg.
  Para mejorar la estabilidad se emplea una segunda retroalimentación, como se muestra en la
  Figura 3(b).
  Determine el valor de \texorpdfstring{$K_h$}{Kh}, para que el factor de amortiguamiento relativo del sistema sea ahora
  0,5. Obtenga la curva de respuesta al escalón unitario (analítica y computacionalmente), tanto
  del sistema original como del sistema con la nueva retroalimentación. Y analice.}

\begin{figure}[h]
	\centering
	\includegraphics[width=0.8\textwidth]{./figures/Figura 3 ejercicio 5.png}
	\caption{Sistema de la Figura 3(a) y 3(b)}
\end{figure}


\subsection{El sistema dado en la Figura 3(a) tiene la siguiente función de transferencia:}

\[
	G(s) = \frac{10}{s(s + 1)}
\]

Con una retroalimentación unitaria, la función de transferencia de lazo cerrado es:

\[
	T(s) = \frac{G(s)}{1 + G(s)} = \frac{\frac{10}{s(s+1)}}{1 + \frac{10}{s(s+1)}} = \frac{10}{s^2 + s + 10}
\]

De acuerdo a los datos proporcionados:

\begin{itemize}
	\item El factor de amortiguamiento relativo (\(\zeta\)) es \(0.158\).
	\item La frecuencia natural no amortiguada (\(\omega_n\)) es \(3.16 \, \text{rad/seg}\).
\end{itemize}

Estos parámetros corresponden a una ecuación característica de segundo orden de la forma:

\[
	s^2 + 2\zeta \omega_n s + \omega_n^2 = 0
\]

Sustituyendo los valores:

\[
	s^2 + 2(0.158)(3.16)s + (3.16)^2 = s^2 + 0.998s + 9.9856
\]

Lo cual se aproxima bastante a la ecuación característica del sistema de lazo cerrado original: \(s^2 + s + 10\).


Implementación computacional en Python
\begin{lstlisting}[language=Python]
	import control as ctrl
	import matplotlib.pyplot as plt
	
	# Sistema original
	num_orig = [10]
	den_orig = [1, 1, 10]
	sys_orig = ctrl.TransferFunction(num_orig, den_orig)

	# Simulacion de la respuesta al escalon
	t, y_orig = ctrl.step_response(sys_orig)

	# Graficar resultados
	plt.plot(t, y_orig, color='b', label='Sistema Original')
	plt.title('Respuesta al escalon unitario')
	plt.xlabel('Tiempo (s)')
	plt.ylabel('Respuesta')
	plt.legend()
	plt.grid()
	plt.show()
\end{lstlisting}

\begin{figure}[h]
	\centering
	\includegraphics[width=0.7\textwidth]{./figures/Figura 4 ejercicio 5.png}
	\caption{Respuesta al escalón unitario del sistema original}
\end{figure}

\subsection{Sistema con nueva retroalimentación (Figura 3b)}

En la Figura 3(b), se añade una segunda retroalimentación con ganancia \(K_h\). Esto modifica la ecuación característica del sistema.

La nueva función de transferencia con \(K_h\) será:

\[
	T_{\text{nuevo}}(s) = \frac{10}{s(s+1) + K_h}
\]

El nuevo polinomio característico del sistema en lazo cerrado es:

\[
	s^2 + s + 10K_h = 0
\]

Comparando el polinomio característico del sistema con la forma estándar:
\[
	s^2 + s + 10K_h = s^2 + 2\zeta \omega_n s + \omega_n^2
\]
De esta comparación, obtenemos dos ecuaciones:
\begin{itemize}
	\item Para el término de \( s \): \( 2\zeta \omega_n = 1 \)
	\item Para el término constante: \( \omega_n^2 = 10K_h \)
\end{itemize}

\subsection*{Cálculo de \( \omega_n \)}
Usamos la primera ecuación \( 2\zeta \omega_n = 1 \) para despejar \( \omega_n \):
\[
	\omega_n = \frac{1}{2\zeta} = \frac{1}{2(0.5)} = 1
\]

\subsection*{Cálculo de \( K_h \)}
Sustituimos el valor de \( \omega_n^2 = 1^2 = 1 \) en la segunda ecuación \( \omega_n^2 = 10K_h \):
\[
	1 = 10K_h
\]
\[
	K_h = \frac{1}{10} = 0.1
\]


Implementación computacional en Python
\begin{lstlisting}[language=Python]
	import control as ctrl
	import matplotlib.pyplot as plt

	# Sistema con nueva retroalimentacion
	K_h = 0.1
	num_new = [10]
	den_new = [1, 1, 10*K_h]
	sys_new = ctrl.TransferFunction(num_new, den_new)

	t2, y_new = ctrl.step_response(sys_new)

	# Graficar resultados
	plt.plot(t2, y_new, color='r', label='Sistema con nueva retroalimentacion')
	plt.title('Respuesta al escalon unitario')
	plt.xlabel('Tiempo (s)')
	plt.ylabel('Respuesta')
	plt.legend()
	plt.grid()
	plt.show()
\end{lstlisting}

\begin{figure}[h]
	\centering
	\includegraphics[width=0.7\textwidth]{./figures/Figura 5 ejercicio 5.png}
	\caption{Respuesta al escalón unitario del sistema con nueva retroalimentación}
\end{figure}

\newpage

\subsection{Respuesta al escalón unitario}

\subsection*{Sistema original}

La función de transferencia del sistema original es:

\[
	T(s) = \frac{10}{s^2 + s + 10}
\]

Usando la transformada inversa de Laplace para la respuesta al escalón unitario, obtenemos una respuesta típica de un sistema de segundo orden subamortiguado, con un factor de amortiguamiento \(\zeta = 0.158\).

\subsection*{Sistema con retroalimentación}

Para el sistema con la segunda retroalimentación, la función de transferencia se convierte en:

\[
	T_{\text{nuevo}}(s) = \frac{10}{s^2 + s + 1}
\]

Este sistema tendrá una mejor estabilidad debido a su mayor factor de amortiguamiento (\(\zeta = 0.5\)), lo que reducirá las oscilaciones en la respuesta al escalón.

\subsection*{Análisis}

El nuevo sistema con \(K_h = 0.1\) tendrá una respuesta al escalón más rápida y con menos sobreoscilación en comparación con el sistema original, que tiene un amortiguamiento mucho menor. La mejora en la estabilidad es evidente al aumentar el factor de amortiguamiento de \(0.158\) a \(0.5\).

Implementación computacional en Python
\begin{lstlisting}[language=Python]
	import control as ctrl
	import matplotlib.pyplot as plt
	
	# Sistema original
	num_orig = [10]
	den_orig = [1, 1, 10]
	sys_orig = ctrl.TransferFunction(num_orig, den_orig)
	
	# Sistema con nueva retroalimentacion
	K_h = 0.1
	num_new = [10]
	den_new = [1, 1, 10*K_h]
	sys_new = ctrl.TransferFunction(num_new, den_new)
	
	# Simulacion de la respuesta al escalon
	t, y_orig = ctrl.step_response(sys_orig)
	t2, y_new = ctrl.step_response(sys_new)

	# Graficar resultados
	plt.plot(t, y_orig, color='b', label='Sistema Original')
	plt.plot(t2, y_new, color='r', label='Sistema con nueva retroalimentacion')
	plt.title('Respuesta al escalon unitario')
	plt.xlabel('Tiempo (s)')
	plt.ylabel('Respuesta')
	plt.legend()
	plt.grid()
	plt.show()
\end{lstlisting}

\begin{figure}[h]
	\centering
	\includegraphics[width=0.7\textwidth]{./figures/Figura 6 ejercicio 5.png}
	\caption{Respuesta al escalón unitario de ambos sistemas}
\end{figure}

\newpage

\section{Problema 6: Considerando el sistema de la Figura 7, determine el valor de k , de modo que el factor de
  amortiguamiento sea 0,5. Basado en esto, obtenga el tiempo de levantamiento tr , el tiempo
  peak tp , el sobrepaso máximo Mp , y el tiempo de asentamiento ts , en la respuesta al escalón
  unitario. Obtenga analíticamente y muestre la respuesta en el tiempo (computacionalmente).}

\begin{figure}[h]
	\centering
	\includegraphics[width=0.8\textwidth]{./figures/Figura 7 ejercicio 6.png}
	\caption{Sistema de la Figura 7}
\end{figure}


La función de transferencia del sistema en la Figura 7 es:

\[
	G(s) = \frac{16}{s + 0.8}
\]

La segunda retroalimentación proporcional tiene una ganancia \(k\). Para encontrar la función de transferencia total en lazo cerrado, debemos considerar ambas retroalimentaciones. La ecuación característica del sistema en lazo cerrado es:

\[
	T(s) = \frac{G(s)}{1 + G(s) \cdot k \cdot \frac{1}{s}} = \frac{\frac{16}{s + 0.8}}{1 + \frac{16k}{s(s + 0.8)}} = \frac{16}{s(s + 0.8) + 16k}
\]

\subsection{Determinación del valor de \texorpdfstring{\(k\)}{k}}

Para un sistema de segundo orden, la ecuación característica estándar es:

\[
	s^2 + 2\zeta \omega_n s + \omega_n^2 = 0
\]

Donde:
\begin{itemize}
	\item \(\zeta = 0.5\) (factor de amortiguamiento relativo),
	\item \(\omega_n\) es la frecuencia natural no amortiguada.
\end{itemize}

La ecuación característica del sistema será:

\[
	s^2 + 0.8s + 16k = 0
\]

El factor de amortiguamiento está relacionado con la frecuencia natural y la constante del término lineal en \(s\) por la fórmula:

\[
	2\zeta \omega_n = 0.8
\]

Sustituyendo \(\zeta = 0.5\):

\[
	2(0.5) \omega_n = 0.8 \quad \Rightarrow \quad \omega_n = \frac{0.8}{1} = 0.8
\]

La frecuencia natural \(\omega_n\) también está relacionada con el término constante en la ecuación característica:

\[
	\omega_n^2 = 16k
\]

Sustituyendo \(\omega_n = 0.8\):

\[
	0.8^2 = 16k \quad \Rightarrow \quad k = \frac{0.64}{16} = 0.04
\]

\subsection{Cálculo de parámetros de la respuesta al escalón unitario}

\subsubsection{Tiempo de levantamiento \texorpdfstring{\(t_r\)}{tr}}

El tiempo de levantamiento es el tiempo que tarda la salida en pasar del 10\% al 90\% del valor final. Para un sistema subamortiguado con \(\zeta = 0.5\), una aproximación es:

\[
	t_r \approx \frac{1.8}{\omega_n} = \frac{1.8}{0.8} = 2.25 \, \text{segundos}
\]

\subsubsection{Tiempo peak \texorpdfstring{\(t_p\)}{tp}}

El tiempo peak es el tiempo que tarda la respuesta en alcanzar su primer máximo. Para un sistema de segundo orden:

\[
	t_p = \frac{\pi}{\omega_n \sqrt{1 - \zeta^2}} = \frac{\pi}{0.8 \sqrt{1 - 0.5^2}} = \frac{\pi}{0.8 \times 0.866} \approx 4.55 \, \text{segundos}
\]

\subsubsection{Sobrepaso máximo \texorpdfstring{\(M_p\)}{Mp}}

El sobrepaso máximo se puede calcular con la siguiente fórmula para un sistema de segundo orden:

\[
	M_p = e^{\left(\frac{-\pi \zeta}{\sqrt{1 - \zeta^2}}\right)} = e^{\left(\frac{-\pi (0.5)}{\sqrt{1 - (0.5)^2}}\right)} = e^{\left(\frac{-1.57}{0.866}\right)} \approx 16.3\%
\]

\subsubsection{Tiempo de asentamiento \texorpdfstring{\(t_s\)}{ts}}

El tiempo de asentamiento es el tiempo que tarda la respuesta en quedarse dentro de un 2\% del valor final. Para un sistema de segundo orden, el tiempo de asentamiento se aproxima como:

\[
	t_s \approx \frac{4}{\zeta \omega_n} = \frac{4}{0.5 \times 0.8} = 10 \, \text{segundos}
\]

\subsection{Simulación computacional}

Implementación computacional en Python
\begin{lstlisting}[language=Python]
	import control as ctrl
	import matplotlib.pyplot as plt
	
	# Parametros del sistema
	k = 0.04  # Valor de k determinado
	num = [16]
	den = [1, 0.8, 16*k]
	
	# Crear el sistema en lazo cerrado
	sys = ctrl.TransferFunction(num, den)
	
	# Simulacion de la respuesta al escalon
	t, y = ctrl.step_response(sys)
	
	# Graficar la respuesta
	
	plt.plot(t, y, label='Respuesta al escalon unitario')
	plt.title('Respuesta al escalon unitario (k = 0.04)')
	plt.xlabel('Tiempo (s)')
	plt.ylabel('Respuesta')
	plt.axvline(x=2.25, color='red', linestyle='--', label='Tiempo de levantamiento (Tr) 2.25 [s]')
	plt.axvline(x=4.55, color='green', linestyle='--', label='Tiempo peak (Tp) 4.55 [s]')
	plt.axvline(x=10, color='purple', linestyle='--', label='Tiempo de asentamiento (Ts) 10 [s]')
	plt.plot([], [], ' ', label='Sobrepaso 16.3%')
	plt.grid()
	plt.legend() 
	plt.show()
\end{lstlisting}

\begin{figure}[h]
	\centering
	\includegraphics[width=0.7\textwidth]{./figures/Figura 8 ejercicio 6.png}
	\caption{Respuesta al escalón unitario del sistema con \(k = 0.04\)}
\end{figure}


Este código genera la respuesta al escalón unitario y te permite analizar visualmente cómo se comporta el sistema con el valor \(k = 0.04\).

\section{Problema 7: Las Figuras 9 muestran tres sistemas. El sistema I es un sistema de control de posición; el
  sistema II es un sistema de control de posición con un control PD; y el sistema III es un sistema
  de control de posición con retroalimentación de velocidad.
  Compare las respuestas a un escalón unitario e impulso unitario de los tres sistemas (computacionalmente, y ojalá en una misma gráfica). ¿Qué sistema es mejor con respecto a la velocidad
  de respuesta, y sobre el sobrepaso (elongación) máxima, en la respuesta al escalón? }

\begin{figure}[h]
	\centering
	\includegraphics[width=0.8\textwidth]{./figures/Figura 9 ejercicio 7.png}
	\caption{Sistemas I, II y III}
\end{figure}

\end{document}